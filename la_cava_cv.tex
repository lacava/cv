\documentclass[10pt]{article}
%\usepackage{sectsty}
%\sectionfont{\fontsize{12}{14}\selectfont}
\usepackage{array, xcolor, lipsum, bibentry}
\usepackage[margin=3cm]{geometry} 
\definecolor{lightgray}{gray}{0.8}
\newcolumntype{L}{>{\raggedleft}p{0.15\textwidth}}
\newcolumntype{R}{p{0.79\textwidth}}
\newcommand\VRule{\color{lightgray}\vrule width 0.5pt}
\usepackage[backend=bibtex,bibstyle=numeric,maxbibnames=99,url=false,isbn=false,defernumbers=true,sorting=ydnt]{biblatex} %
\DeclareFieldFormat{extrayear}{}
\renewbibmacro{in:}{}
\renewbibmacro{in}{}
\usepackage{tabularx}
\bibliography{La_Cava} 
\renewcommand{\arraystretch}{1.2}\addtolength{\tabcolsep}{.5pt}

%%%
% Make my name bold 
%%%%%%%%%%%%%%%%%%%%%%%%%%%%%%%%%%%%%%%%%%%%%%%%%%%%%%%%%%%%%%%%%%%%%%%%%%%%%%%%%
\usepackage{xpatch}% or use http://tex.stackexchange.com/a/40705

\def\makenamesetup{%
  \def\bibnamedelima{~}%
  \def\bibnamedelimb{ }%
  \def\bibnamedelimc{ }%
  \def\bibnamedelimd{ }%
  \def\bibnamedelimi{ }%
  \def\bibinitperiod{.}%
  \def\bibinitdelim{~}%
  \def\bibinithyphendelim{.-}}    
\newcommand*{\makename}[3]{\begingroup\makenamesetup\xdef#1{#2, #3}\endgroup}

\newbibmacro*{name:bold}[2]{%
  \makename{\currname}{#1}{#2}%
  \makename{\findname}{\lastname}{\firstname}%
  \makename{\findinit}{\lastname}{\firstinit}%
  \ifboolexpr{ test {\ifdefequal{\currname}{\findname}}
            or test {\ifdefequal{\currname}{\findinit}} }{\bfseries}{}}

\newcommand*{\boldname}[3]{%
  \def\lastname{#1}%
  \def\firstname{#2}%
  \def\firstinit{#3}}
\boldname{}{}{}

\xpretobibmacro{name:family}{\begingroup\usebibmacro{name:bold}{#1}{#2}}{}{}
\xpretobibmacro{name:given-family}{\begingroup\usebibmacro{name:bold}{#1}{#2}}{}{}
\xpretobibmacro{name:family-given}{\begingroup\usebibmacro{name:bold}{#1}{#2}}{}{}
\xpretobibmacro{name:delim}{\begingroup\normalfont}{}{}

\xapptobibmacro{name:family}{\endgroup}{}{}
\xapptobibmacro{name:given-family}{\endgroup}{}{}
\xapptobibmacro{name:family-given}{\endgroup}{}{}
\xapptobibmacro{name:delim}{\endgroup}{}{}

\boldname{La~Cava}{William}{W}
%%%%%%%%%%%%%%%%%%%%%%%%%%%%%%%%%%%%%%%%%%%%%%%%%%%%%%%%%%%%%%%%%%%%%%%%%%%%%%%%%
%%%
\usepackage{hyperref}
\hypersetup{colorlinks=true,
	citecolor=blue,
	linkcolor=blue,
	urlcolor=blue
}

\begin{document}
\noindent \begin{tabularx}{\textwidth}{ll}
%\hspace{0.5\textwidth}
{\large \bf William La Cava} \\ \hline
{\it \today} \\
Office & Richards Building D207-04 \\
&3700 Hamilton Walk \\ 
&Philadelphia, PA 19104 \\ 
Email & \texttt{lacava@upenn.edu} \\
Website & \href{http://williamlacava.com}{\texttt{williamlacava.com}} \\ 
%122 Hawley St.\\Northampton, MA 01060\\$+$1 (413) 320-0544\\wlacava@umass.edu\\

\end{tabularx}

 
\section*{Education}
\begin{tabular}{L!{\VRule}R}
2012 - 2016& Ph.D., Mechanical Engineering, University of Massachusetts Amherst\\
2009 - 2010& M.Eng., Mechanical Engineering, Cornell University \\
2005 - 2009& B.S., Mechanical Engineering, Cornell University 
\end{tabular}
 
\section*{Research Interests}
Biomedical informatics, 
machine learning, 
data science, 
genetic and evolutionary computation,  
dynamical systems, 
wind energy, 
artificial intelligence and artificial life
 
\section*{Research Experience}

\begin{tabular}{L!{\VRule}R}
2019 -  & {\bf Consultant, National Institute on Aging, National Institutes of Health} \\
& Division of Geriatrics and Clinical Gerontology \\
2016 -  & {\bf Postdoctoral Fellow, Epistasis Laboratory, University of Pennsylvania} \\
& {\it Advisor: Jason Moore} \\
& Institute for Biomedical Informatics \\
& Fellow, Warren Center for Network and Data Sciences \\
2012 - 2016 & {\bf PhD Student, University of Massachusetts Amherst} \\
& {\it Committee: Kourosh Danai, Lee Spector, Matthew Lackner} \\
& Fellow, NSF IGERT Offshore Wind Energy Program \\
%I research local search methods for topological search of model structures. I have developed both gradient-based and symbolic regression methods for formulating succinct and accurate models of complex systems. As an NSF IGERT fellow studying offshore wind energy, I have applied these methods to the identification of new models of wind turbine dynamics, vortex-induced vibration, and bald eagle behavior around wind farms, in addition to providing software to other researchers at UMass. \\ 
Jun--Aug 2015& {\bf Visiting Researcher, Laboratory of Agent Modeling, University of Lisbon} \\
&{\it Hosts: Sara Silva, Leonardo Vanneschi} \\
& Subject: Multiclass classification of complex systems using genetic programming \\
%, with applications to agent-based modeling of avian behavior around wind farms \\
%&I collaborated with Professors Silva and Vanneschi on new methods of multiclass classification using genetic programming. \\
2010 - 2012& {\bf Research Scientist, National Renewable Energy Laboratory (NREL)} \\
& {\it Supervisors: Paul Veers, Jonathan Keller} \\
& Lead engineer for the Gearbox Reliability Collaborative, a consortium involved in wind turbine gearbox testing, data analysis, and numerical modeling \\
& Designed and conducted drivetrain simulation and testing programs for a 3 MW wind turbine R\&D project \\
%& I led the modeling and analysis team of the Gearbox Reliability Collaborative (GRC). Work involved wind turbine gearbox testing, data analysis, and numerical modeling. I collaborated with industry partners to study gearbox reliability issues. I also lead a drivetrain simulation and testing program for a 3 MW wind turbine R\&D project.\\
\end{tabular}

\noindent \begin{tabular}{L!{\VRule}R}
2008 - 2010& {\bf Lead Engineer of Mechanical Power Systems, Cornell 100+ MPG Team} \\
& {\it Advisor: Albert George} \\
& Design, fabrication and testing for a hybrid-electric vehicle that competed in the Automotive X-Prize and won the 2011 Green Grand Prix, achieving over 120 MPG equivalent\\
%& I oversaw a team of 7 engineers in drivetrain design, fabrication and testing for a hybrid-electric vehicle that competed in the Automotive X-Prize. Designed and fabricated an electric drivetrain, a regenerative braking system, and a turbo-diesel charging system. The four-person sedan took first prize at the Green Grand Prix competition in 2011, achieving over 120 MPG. \\
2007 - 2008& {\bf Independent Research, Cornell Computational Synthesis Laboratory} \\
& {\it Advisor: Hod Lipson}\\
&Built and trained a mobile robot with a 5 degree-of-freedom arm and gripper to retrieve objects \\
\end{tabular}
 
\section*{Grants \& Awards}
\begin{tabular}{L!{\VRule}R}
    2019& {\bf NIH Pathway to Independence Award (K99/R00) }\\
        & National Library of Medicine\\
        & Title: Multi-objective representation learning methods for interpretable predictions of patient outcomes using electronic health records 
    (\href{https://projectreporter.nih.gov/project_info_description.cfm?aid=9744166&icde=0}{link})\\
2019& Best Paper Nomination, Genetic and Evolutionary Computation Conference  \\
2019& Winner, Best Informatics Abstract, DBEI and CCEB Research Day\\
2018& Best Paper Nomination, Genetic and Evolutionary Computation Conference  \\
2017& Best Paper Nomination, European Conference on Genetic Programming  \\
2016& Postdoctoral Fellowship, Warren Center for Network and Data Sciences \\
2016& Student Travel Grant, Genetic and Evolutionary Computation Conference\\
2015& Student Travel Grant, ASME Dynamic Systems and Controls Conference \\
2015& Best Paper Nomination, Genetic and Evolutionary Computation Conference  \\
2014&XSEDE Startup Allocation Award: Automatic Identification of Dynamic Models for Complex Systems (PI) \\
2012& NSF Fellowship, IGERT: Offshore Wind Energy Engineering, Environmental Science, and Policy \\
2011& First Place, Cornell 100$+$ MPG Team, Green Grand Prix Competition \\
\end{tabular}

\section*{Publications}
\small{\href{https://scholar.google.com/citations?user=iZB7inEAAAAJ&hl=en}{Google Scholar} (citations: 975, h-index: 15, i10-index: 27) }

\printbibliography[title={\normalsize Articles in Review},keyword=inreview,resetnumbers=true]
\nocite{*}
% \printbibliography[title={\normalsize Refereed Publications},keyword=peerreviewed,notkeyword=inreview,resetnumbers=true]
% \nocite{*}
\printbibliography[title={\normalsize Journal Papers},type=article,notkeyword=inreview,resetnumbers=true]
\nocite{*}
\printbibliography[title={\normalsize Peer-reviewed Conference Proceedings},type=inproceedings,notkeyword=inreview,resetnumbers=true]
\nocite{*}
\printbibliography[title={\normalsize Book Chapters},type=incollection,notkeyword=inreview,notkeyword=thesis,resetnumbers=true]
\nocite{*}
\printbibliography[title={\normalsize Dissertations},type=incollection,keyword=thesis,notkeyword=inreview,resetnumbers=true]
\nocite{*}
\printbibliography[title={\normalsize Technical Reports},type=report,notkeyword=inreview,resetnumbers=true]
\nocite{*}
\printbibliography[title={\normalsize Press},keyword=press,notkeyword=inreview,notkeyword=thesis,resetnumbers=true]
\nocite{*}
\printbibliography[title={\normalsize Software},keyword=software,notkeyword=inreview,resetnumbers=true]
\nocite{*}
\printbibliography[title={\normalsize Video},keyword=video,notkeyword=inreview,resetnumbers=true]
\nocite{*}
\printbibliography[title={\normalsize Invited Talks},keyword=invited,notkeyword=inreview,resetnumbers=true]
\nocite{*}
% \printbibliography[title={\normalsize Other Presentations},keyword=presentation,notkeyword=inreview,resetnumbers=true]
% \nocite{*}
% \printbibliography[title={\normalsize Other Conference Posters},keyword=poster,notkeyword=inreview,resetnumbers=true]
% \nocite{*}

\section*{Teaching Experience}
\begin{tabular}{L!{\VRule}R}
2017 - & {\bf Guest Lecturer, University of Pennsylvania}\\
& Courses: Data Science for Biomedical Informatics; Special Topics in Biomedical and Health Informatics \\
& Topics: Supervised and unsupervised machine learning; nature-inspired computing \\
2014 - 2016& {\bf Guest Lecturer, University of Massachusetts Amherst}\\
& Courses: System Dynamics; Control Systems Laboratory; Offshore Wind Energy Design\\
& Topics: linearization; state-space representations; system identification; parameter estimation; and wind turbine control design \\

2014 - 2015& {\bf Teaching Assistant, University of Massachusetts Amherst} \\
& Control Systems Laboratory \\
2007& {\bf Lab Technician, Cornell University}\\
& Designed and built robotic platforms for a graduate level artificial intelligence course\\
\end{tabular}

\section*{Mentoring Experience}
\begin{tabular}{L!{\VRule}R}
Jun 2017 - & {\bf Research Mentor, University of Pennsylvania}\\
           & Students: 
                Isabel Lee (B.S.);
                Efe Ayhan (B.S.);
                Max Roling (B.S.); 
                Saurav Bose (M.S.);
                Tilak Raj Singh (M.S.);
                Rishabh Gupta (M.S.);
                Sophia Moses (B.S.);
                James Taggart (B.S.);
                Srinivas Suri (M.S.)
                \\
                Jun - Aug 2015& {\bf NSF Research Experiences for Undergraduates (REU) Mentor, University of Massachusetts Amherst} \\
& Student: Branch Vincent (B.S.) \\
\end{tabular}

\section*{Service}
\begin{tabular}{L!{\VRule}R}
    Organizer 
        & Good Benchmarking Practices for Evolutionary Computation, GECCO and PPSN Workshop (2020) \\
        & New Standards for Benchmarking in Evolutionary Computation Research, GECCO Workshop (2017-2019) \\
        & Collaboration with University of Maine's Advanced Structures and Composites Center (2014) \\
        & Gearbox Reliability Collaborative Annual Meeting, National Renewable Energy Laboratory (2011, 2012) \\
    Committee Member 
        & International Workshop on Benchmarking of Computational Intelligence Algorithms, ICACI (2018) \\
    Member 
        & Association of Computing Machinery (ACM) \\
        & International Society for Computational Biology (ISCB) \\
        & American Society of Mechanical Engineers (ASME) \\
        & American Institute of Aeronautics and Astronautics (AIAA) \\
    Referee 
        & Pacific Symposium on Biocomputing\\
        & PLOS ONE \\
        & Artificial Life Journal \\
        & Genetic and Evolutionary Computation Conference \\
        & European Conference on Genetic Programming \\
        & Genetic Programming and Evolvable Machines \\
        & IEEE Transactions on Neural Networks and Learning Systems\\
        & IEEE Congress on Evolutionary Computation \\
        & Swarm and Evolutionary Computation \\
        & Information Journal \\
        & Wind Energy Journal \\
        & Renewable Energy Journal \\
        & AIAA Wind Energy Symposium (2014) \\
        & ASME Dynamic Systems and Controls Conference (2015) \\
\end{tabular}

\section*{Volunteer \& Outreach Activities}
\begin{tabular}{L!{\VRule}R}
2019& Green Labs participant, Perelman School of Medicine\\
2019& Volunteer, Love Your Park Week, Fairmount Park Conservancy\\
2016& {\bf Science Fair Judge, Hampshire Regional High School} \\
2013 - 2014& {\bf Invited Science Teacher, Four Rivers Charter School}\\
& Taught classes on wind energy to high school students\\
2011 - 2012& {\bf Volunteer, Boulder Food Rescue}\\
& This organization has saved hundreds of thousands of pounds of left over food from grocery stores and bakeries and delivered it to homeless shelters and other community food stations. \\
%2005 - 2010& Vice President Fanclub Collective. Music promotion agency in Ithaca, NY \\
2001 - 2005& {\bf American Cancer Society Relay for Life}\\
\end{tabular}

\section*{Other Interests}
\begin{tabular}{L!{\VRule}R}
Film & I write, direct, and produce short fictional films, including: \\
& ``MADG" (2014), {\it Sound on Sound Film Festival} (premiere), {\it Florence Night Out} \\
& ``Vacuumland Trilogy" (2008), {\it The Project TV Competition} \\
Music & VP, Fanclub Collective, a music promotion agency in Ithaca, NY (2005 - 2010)\\
& I write, record, and produce music \\
Language & Spanish (advanced), Italian (beginner) \\
Sports & Muay Thai, rock climbing, soccer
\end{tabular}
\end{document}
