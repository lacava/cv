\section*{Research Interests}
Biomedical informatics, 
machine learning, 
data science, 
fairness and interpretability,
genetic and evolutionary computation,  
dynamical systems, 
wind energy, 
artificial intelligence and 
artificial life
 
\section*{Research Experience}

\begin{tabular}{L!{\VRule}R}
2020 - & {\bf Research Associate,  University of Pennsylvania} \\
& Institute for Biomedical Informatics \\
2019 - 2020  & {\bf Consultant, National Institute on Aging, National Institutes of Health} \\
& Division of Geriatrics and Clinical Gerontology \\
2016 - 2020 & {\bf Postdoctoral Fellow, Epistasis Laboratory, University of Pennsylvania} \\
& {\it Advisor: Jason Moore} \\
& Institute for Biomedical Informatics \\
& Fellow, Warren Center for Network and Data Sciences \\
2012 - 2016 & {\bf PhD Student, University of Massachusetts Amherst} \\
& {\it Committee: Kourosh Danai, Lee Spector, Matthew Lackner} \\
& Fellow, NSF IGERT Offshore Wind Energy Program \\
%I research local search methods for topological search of model structures. I have developed both gradient-based and symbolic regression methods for formulating succinct and accurate models of complex systems. As an NSF IGERT fellow studying offshore wind energy, I have applied these methods to the identification of new models of wind turbine dynamics, vortex-induced vibration, and bald eagle behavior around wind farms, in addition to providing software to other researchers at UMass. \\ 
Jun--Aug 2015& {\bf Visiting Researcher, Laboratory of Agent Modeling, University of Lisbon} \\
&{\it Hosts: Sara Silva, Leonardo Vanneschi} \\
& Subject: Multiclass classification of complex systems using genetic programming \\
%, with applications to agent-based modeling of avian behavior around wind farms \\
%&I collaborated with Professors Silva and Vanneschi on new methods of multiclass classification using genetic programming. \\
2010 - 2012& {\bf Research Scientist, National Renewable Energy Laboratory (NREL)} \\
& {\it Supervisors: Paul Veers, Jonathan Keller} \\
& Wind turbine field testing, numerical modeling, and data analysis
% & Lead engineer for the Gearbox Reliability Collaborative, a consortium involved in wind turbine gearbox testing, data analysis, and numerical modeling \\
% & Designed and conducted drivetrain simulation and testing programs for a 3 MW wind turbine R\&D project \\
%& I led the modeling and analysis team of the Gearbox Reliability Collaborative (GRC). Work involved wind turbine gearbox testing, data analysis, and numerical modeling. I collaborated with industry partners to study gearbox reliability issues. I also lead a drivetrain simulation and testing program for a 3 MW wind turbine R\&D project.\\
\end{tabular}

\noindent \begin{tabular}{L!{\VRule}R}
2008 - 2010& {\bf Lead Engineer of Mechanical Power Systems, Cornell 100+ MPG Team} \\
& {\it Advisor: Albert George} \\
& Ddrivetrain esign, fabrication and testing for a hybrid-electric vehicle that competed in the Automotive X-Prize and won the 2011 Green Grand Prix, achieving over 120 MPG equivalent\\
%& I oversaw a team of 7 engineers in drivetrain design, fabrication and testing for a hybrid-electric vehicle that competed in the Automotive X-Prize. Designed and fabricated an electric drivetrain, a regenerative braking system, and a turbo-diesel charging system. The four-person sedan took first prize at the Green Grand Prix competition in 2011, achieving over 120 MPG. \\
2007 - 2008& {\bf Independent Research, Cornell Computational Synthesis Laboratory} \\
& {\it Advisor: Hod Lipson}\\
&Robotics, path planning and artificial intelligence \\
\end{tabular}
